\documentclass{article}
\usepackage{fontspec}
\usepackage{jim,amsmath}
\usepackage[french]{babel}
\usepackage{listings}
\usepackage{diagbox}
\usepackage{caption}
\usepackage{booktabs}
\usepackage{microtype}
%\usepackage{pxfonts}
\usepackage{graphicx}
\usepackage{hyperref}
\usepackage{tikz}

\newcommand\vocab[1]{\emph{#1}}
\DeclareCaptionType{scenario}[Scénario]
% Title.
% ------
\title{Écriture et exécution répartie de scénarios interactifs}

\threeauthors
  {Auteur 1} {Organisme \\ Adresse électronique}
  {Auteur 2} {Organisme \\ Adresse électronique}
  {Auteur 3} {Organisme \\ Adresse électronique}

\begin{document}
\maketitle
\begin{abstract}
Le résumé doit être placé en haut de la colonne gauche et doit contenir entre 150 et 200 mots.
\end{abstract}
\section{Introduction}
\subsection{Problématique}

\subsection{Études de cas}
-> installation en son réparti ?

- Problème de la latence entre i-score et raspberry, + utilisation de bande passante.

- Cas des applis de téléphone : un objet qui s'exécute sur plusieurs machines en parallèle dont on veut aggréger les résultats

- Donner la possibilité de faire un dialogue (boucle avec un coup un trigger qui vient de A, un coup un trigger qui vient de B, etc)

- Approche alternative avec application mobile.

- Faciliter l'écriture de programmes qui s'exécutent sur des machines embarquées

- Lister les cas. D'abord non interactif (partage uniquement des processus), puis interactif.

\subsection{État de l'art}
\subsubsection{Rappel du modèle d'i-score}
Vocabulaire. Prendre article SMC.
%\subsubsection{Répartition à l'édition}
\subsubsection{Répartition à l'exécution}
%\subsubsection{Lien entre la répartition à l'édition et la répartition à l'exécution}

- Problème des devices. Note : si on se permet d'envoyer des messages OSC quelconques (i.e. pas dans l'arbre), ça peut simplifier des choses. Aussi, permettre devices qui sont juste en mode envoi ? Il faut un autre proto que OSC ou Minuit...

Aussi, refactoriser pour utiliser tout i-score en librairie

\subsubsection{Horloges}
Problème de l'horloge : 
\begin{itemize}
\item Si on utilise l'horloge système, pas de garantie qu'elle soit bien synchronisée. 
Et NTP pas dispo partout (on n'a pas forcément les droits sur un téléphone pour changer l'heure).
Horloge i-score se resynchronise déjà en cas de délai sur une machine.
\item Si on utilise une horloge interne, problème de la synchronisation avec l'horloge système (pour le tick)
\end{itemize}

\url{http://queue.acm.org/detail.cfm?id=2745385}
\url{http://radar.oreilly.com/2012/10/google-spanner-relational-database.html}
\url{http://www.ntp.org/ntpfaq/NTP-s-sw-clocks-quality.htm}

Faire la relation entre notre système et le problème CAP : \url{https://en.wikipedia.org/wiki/PACELC_theorem}

Horloges logiques : équivalent à envoyer une timestamp à chaque tick (on arrive donc à un méchanisme de vector clock approximatif). 

Synchro d'horloges de deux machines dont une qui ralenti, si elles exécutent le même 
scénario ? Il faut les recaler régulièrement. 

Possibilité : externe (NTP, etc) ou interne : \url{https://github.com/ethanlim/NetworkTimeProtocol}

Recaler dès qu'on accumule du retard ?



On peut introduire un recalage sur la master clock entre chaque tick.

Ce qu'on fait : ping régulier vers chaque client (toutes les 100 millisecondes)

Quand quelque chose doit se synchroniser, on dit à chaque machine à quel instant il est supposé arriver par rapport à son horloge système.

Quand un client reçoit un ordre pour un timenode à t, il l'applique dès que t <= local(t) (modulo un tick?)



Possibilité : synchronisation via démon externe (PTP, NTP...), mais pas toujours possible (on ne peut pas supposer que l'utilisateur a les droits pour changer l'horloge sur sa machine).

Synchronous ethernet

Ableton Link : synchro sur les ticks musicaux 

Avoir une horloge propre à i-score ? Mais du coup maintenant il faut la synchroniser à l'horloge système. 

\subsubsection{Problématique de la sécurité}

\subsubsection{Synchronisation de médias}
- pas pour l'instant... doivent être au même endroit sur la même machine.
Ableton Link ? Netjack ?

\section{Approche}
Cette section détaille les choix de haut niveau réalisés.

On souhaite modifier le moins possible le modèle pour les utilisateurs du logiciel, 
en rajoutant les notions nécessaires et suffisantes pour offrir la finesse de répartition désirée.

La section~\ref{sec.description} présente de manière détaillée les possibilités 
de répartition que l'on offre.

La section~\ref{sec.semantique} définit ces possibilités en utilisant les objets du modèle d'i-score.

Enfin, la section~\ref{sec.evaluation} présente une évaluation des performances dans le cadre 
d'une implémentation pratique.

\subsection{Nouvelles notions}
Nous nous trouvons en présence de plusieurs machines qui communiquent et partagent un document.
L'ensemble constitué par les instances d'i-score et le document qu'elles partagent est appelé \vocab{session}.

On désigne par \vocab{client} une instance d'i-score connectée à une << session >>.

On désire s'affranchir des notions propres aux machines physiques et des problématiques de réseau (addresse IP, etc) lors de l'écriture 
d'un scénario réparti. 

Pour ce faire, on introduit la notion de \vocab{groupe}. 
Un groupe est un ensemble de clients.

Les compositeurs ne travaillent jamais directement avec un client, mais toujours au travers d'un groupe qui peut contenir zéro, un, ou plusieurs clients.

De manière générale, quand plusieurs clients font partie d'un même groupe, cela signifie qu'ils vont réaliser les mêmes tâches, de manière synchrone ou asynchrone.

L'intérêt d'un groupe est la tolérance aux pannes, déconnections, reconnections, et changements d'installation. 
Par exemple, si une machine tombe en panne, on peut la remplacer par une autre simplement en l'assignant au même groupe que la machine en panne 
sans avoir besoin de mettre à jour le scénario.

\subsection{Topologie}
Maître-esclave pour l'édition.

Pour des raisons de simplicité, la première mise en \oe uvre se fait via une toplogie réseau en étoile : un maître gère 
le déroulement général de l'exécution. 
Les différents clients communiquent au travers de ce maître, avec les avantages et inconvénients que cela implique : 
\begin{itemize}
	\item Facilité d'analyse lors du développement : on peut enregistrer tous les messages échangés avec leurs estampilles.
	\item Intolérance aux pannes : si le maître faillit, aucune récupération n'est possible.
\end{itemize}

On détaille dans les sections suivantes les cas ou des topologies différentes peuvent montrer un intérêt.

Pour l'exécution : dans certains cas deux machines peuvent échanger directement.
Il faut décrire ces cas.

Si groupes avec plusieurs machines : "group leader" ? responsable de la communication ? avoir un "plus court chemin" permanent ?


\subsection{Fondements}
Première approche basées sur réseau de Petri.

Puis migration du modèle d'i-score maintenant autonome.

Il est possible d'implément ce méchanisme de répartition purement avec les primitives dont on dispose dans i-score, 
mais c'est prohibitoire pour les utilisateurs.

Différentes possibilités de répartition sont disponibles selon les types de processus.

Détailler les implications des choix par rapport au théorême CAP et PACELT.
\subsection{Modèle}
i-score se base sur une approche orientée document.

On choisit de répartir nos objets autour d'un même document partagé par toutes les instances du réseau, 
à la manière des logiciels d'édition tels que Google Docs. %CITE

Cela ne pose pas de problème de mémoire, même sur de l'embarqué : 
les scénarios i-score sont très compacts (en général quelques dizaines de kilo-octets).

\subsection{Répartition de l'édition}
- Partage de file undo - redo. Transformations opérationnelles. %CITE

- Autres approches : partage du modèle lui-même.

\section{Description de l'exécution}\label{sec.description}
\subsection{Processus de contenu}
Automation, mapping, JS, audio...

Le processus s'exécute tel quel pour toutes les machines auquel il est assigné, 
et ne s'exécute pas pour les autres.

\subsection{Processus Scenario}
\subsubsection{Cas interactif}
Trois niveaux de partage : 

\begin{itemize}
    \item Aucun partage : les machines n'étant pas associées à ce processus ne l'exécutent pas, celles qui y sont associées l'exécutent toutes de manière indépendante. Note : scénarios hiérarchiques ? 
    \item Partage partiel : il peut y avoir plusieurs lignes temporelles dans un même scénario qui doivent se resynchroniser. Les annotations donnent l'emplacement d'exécution, et de vérification des expressions.
    \item Partage complet : il n'y a qu'une seule "ligne temporelle" pour toutes les machines. Les annotations de machines indiquent l'emplacement d'exécution.
\end{itemize}

Deux niveaux de synchronisation : 
\begin{itemize}
    \item Asynchrone : dès qu'une information est disponible dans le système (par exemple << une expression se vérifie >>), elle est propagée le plus vite possible aux autres noeuds qui doivent appliquer le résultat de cette information.
    \item Synchrone : lorsqu'une information est disponible dans le système, elle est propagée de manière à ce que la date absolue de réalisation soit la même pour toutes les machines.
\end{itemize}

Problème transverse : ignorer ou non la hiérarchie ? 

Différence entre synchro partielle et complète : 
\begin{itemize}
    \item Exécution des triggers
    \item Validité des conditions
    \item Changement de vitesse d'exécution des contraintes temporelles
\end{itemize}

Dans tous les cas, cela se fait au niveau des groupes, et il peut y avoir plusieurs machines par groupe.
Donc il faut un méchanisme de décision au niveau du groupe.
-> pourquoi ne pas avoir la même chose au niveau des mappings (et pour tout ce qui n'est pas fixé de manière générale ?)

\subsubsection{Cas non-interactif}
% Cas principalement développé dans rapport de master
On peut soit utiliser les méthodes de synchronisation décrites précédemment, mais une possibilité s'ajoute : 

Comme on connait les dates effectives auxquelles les objets sont sensés s'exécuter, on peut les fixer à l'avance sur chaque machine. 

\begin{itemize}
\item Avantage : s'il y a une déconnection, cela va continuer à marcher.
\item Inconvénient : il suffit d'un peu de délai pour que le début de B peut survenir avant la fin de A. 
Il est particulièrement important dans ce cas de garder les horloges synchronisées.
\end{itemize}

\subsubsection{Expressions et interactivité}
Dans le cas ou un scénario est exécuté intégralement en parallèle par différentes machines, il n'y a pas de problème : chacune vérifie les expressions en fonction des données dont elle dispose, et les valide à l'instant ou elle le souhaite. C'est utile si par exemple on veut avoir plusieurs téléphones qui font tous tourner un scénario semblable, mais chaque individu peut choisir de faire avancer le scénario au rythme ou il le souhaite.

Dans le cas ou on a un partage de certaines lignes temporelles par certaines machines, le problème de la synchronisation des expressions se pose.

\begin{itemize}
	\item Prise de décision : s'il n'y a qu'une seule ligne temporelle, chaque expression doit n'avoir qu'une valeur de vérité. Il doit donc y avoir un consensus sur la valeur de cette expression. De même pour la vitesse d'exécution des éléments.
    
    Plusieurs algorithmes de consensus sont possibles : 
    \begin{itemize}
        \item Au moins une machine valide une expression. 
        Et spécifiquement, dans le cas des points d'interaction : la première machine qui valide une expression donne le résultat.
        \item Toutes les machines valident une expression.
    \end{itemize}

	\item On peut supposer que des sous-parties d'un scénario pourraient être exécutées entièrement par une machine ou groupe.
\end{itemize}
Ordonnancement : par estampilles.

\begin{figure}[h]
    \centering
    \begin{tikzpicture}
    \input{scenarios/trigger-1.tex}
    \end{tikzpicture}
    \captionof{scenario}{Des groupes A et B exécutent les deux contraintes temporelles et un groupe C l'expression}
    \label{scenar.trigger-1}
\end{figure}

Le scénario~\ref{scenar.trigger-1} peut se résoudre de plusieurs manières.

\begin{itemize}
    \item Si l'exécution du noeud est résolue de manière asynchrone, toutes les machines de $C$ envoient l'information de trigger à toutes les machines de $A$ et $B$, sans garantie d'ordre. Les triggers se déclenchent immédiatement.
    
    Note : si on veut garantir un ordre, ne peut-on pas faire : $C$ stoppe $A$ qui démarre $B$ ? Ou bien si la borne max est atteinte, $A$ stoppe $C$ et déclenche $B$.
    
    \item Si l'exécution du noeud est résolue de manière synchrone : 
    \begin{itemize}
        \item Si la condition devient vraie, par consensus. On choisit une date, et on fixe l'arrêt des machines de $A$ et le départ des machines de $B$ par rapport à cette date, calculée en fonction des latences relatives. Potentiellement introduire un tick d'écart entre la fin de $A$ et le démarrage de $B$ ?
        \item Si on atteint la borne max : une machine est nécessairement la première à atteindre cette borne.
    \end{itemize}
\end{itemize}

\subsubsection{Consensus}
Paxos, Raft sont des algorithmes permettant de garantir un consensus.%CITE 
% https://github.com/apache/kudu


S'il y a synchronisation, on sépare la synchronisation du consensus, de la synchronisation de l'exécution qui suit la résolution de ce consensus : 
\begin{enumerate}
    \item Les n\oe uds impliqués dans l'expression décident de la valeur de vérité.
    \item Une fois cette valeur connue, tous les n\oe uds précédant, suivant, et impliquant l'expression sont inclus dans la décision de la date d'exécution.
\end{enumerate}

Note : abstraire la notion de consensus au sein d'un groupe.

Cas avancé et non traité ici : si un scénario est exécuté en parallèle par toutes les machines, chaque groupe peut avoir un consensus différent sur chaque trigger.
On peut rajouter un niveau d'indirection. Le plus général : chaque objet peut être exécuté par N groupes pouvant posséder chacun N clients.
Si un client est dans deux groupes, comment gérer le conflit ? bof.

\subsection{Introduction de primitives réparties dans i-score}
TODO voir politiques d'exécution de réseaux de Petri dans thèse AA

\subsubsection{Répartition des contenus dans le cas non-interactif}
\begin{figure}[h]
    \centering
    \begin{tikzpicture}
    \input{scenarios/simple.tex}
    \end{tikzpicture}
    \captionof{scenario}{Deux contraintes temporelles se suivent}
    \label{scenar.simple}
\end{figure}

\begin{itemize}
    \item Cas partagé : toutes les machines vont exécuter l'ensemble du scénario en parallèle.
    \item Cas synchronisé : chaque machine va exécuter une partie du scénario : par exemple, une machine exécute $C_1$ et une autre $C_2$.
\end{itemize}

On sépare donc les objets structurants, des objets possédant des données (par exemple, une automation).


\subsubsection{Répartition des contenus dans le cas interactif}
\begin{figure}[h]
    \centering
    \begin{tikzpicture}
    \fill (0, 18.7148) circle (0.075) ; % State.0 
\fill (2.15596, 18.7148) circle (0.075) ; % State.1 
\fill (2.15596, 17.4029) circle (0.075) ; % State.2 
\fill (4.40367, 17.4029) circle (0.075) ; % State.3 
\fill (0, 18.0053) circle (0.075) ; % State.4 
\fill (2.15596, 18.0187) circle (0.075) ; % State.5 
\fill (2.15596, 16.7068) circle (0.075) ; % State.6 
\fill (5, 16.7068) circle (0.075) ; % State.7 
\draw[line width=1pt] (0, 18.7148)  -- (0, 18.0053) ; % TimeNode.0 
\draw[line width=1pt] (2.15596, 18.7148)  -- (2.15596, 16.7068) ; % numb47vine94 
\draw (2.15596, 18.9648) node {$C$}; % numb47vine94 
\draw[line width=1pt] (4.40367, 17.4029)  -- (4.40367, 17.4029) ; % faze44greg1 
\draw[line width=1pt] (5, 16.7068)  -- (5, 16.7068) ; % burp79lawn67 
\draw[line width=1pt] (0, 18.7148)  -- (1.39908, 18.7148) ; % dido10rend91 
\draw[dashed,line width=1pt] (1.39908, 18.7148)  -- (2.86239, 18.7148) ; % dido10rend91 
\draw[line width=0.7pt] (1.63908, 18.8678) arc(90:270:0.15) ; % dido10rend91 
\draw[line width=0.7pt] (2.71239, 18.5648) arc(-90:90:0.15) ; % dido10rend91 
\draw (1.07798, 18.9148) node {$A$}; % dido10rend91 
\draw[line width=1pt] (2.15596, 17.4029)  -- (4.40367, 17.4029) ; % taut8hews94 
\draw (3.27982, 17.6029) node {$D$}; % taut8hews94 
\draw[line width=1pt] (0, 18.0053)  -- (1.38073, 18.0053) ; % ramp48lust85 
\draw[dashed,line width=1pt] (1.38073, 18.0053)  -- (2.90826, 18.0053) ; % ramp48lust85 
\draw[line width=0.7pt] (1.62073, 18.1583) arc(90:270:0.15) ; % ramp48lust85 
\draw[line width=0.7pt] (2.75826, 17.8553) arc(-90:90:0.15) ; % ramp48lust85 
\draw (1.07798, 18.2053) node {$B$}; % ramp48lust85 
\draw[line width=1pt] (2.15596, 16.7068)  -- (5, 16.7068) ; % dirt50gage22 
\draw (3.57798, 16.9068) node {$E$}; % dirt50gage22 
\draw[line width=1pt] (1.95596, 16.7068)  -- (1.95596, 16.7068) ; % sill78hues28 
\draw[line width=1pt] (1.95596, 16.7068) arc(180:75:0.2) ; % sill78hues28 
\draw[line width=1pt] (1.95596, 16.7068) arc(180:285:0.2) ; % sill78hues28 

    \end{tikzpicture}
    \captionof{scenario}{Interactions et conditions}
    \label{scenar.general}
\end{figure}

\subsection{Récapitulatif}
On choisit pour la répartition un modèle de document partagé : toutes les machines voient le même scénario.

On introduit une notion de client (une machine physique) et de groupe.
Un groupe peut contenir plusieurs clients, et un client peut être présent dans plusieurs groupes.

Différents objets du modèle d'i-score peuvent être assignés à un groupe : 

\begin{itemize}
	\item Les processus
	\item Les expressions
	\item Les contraintes temporelles
\end{itemize}

Donner un groupe à une contrainte temporelle propage ce groupe à tous ses processus enfants.

Pour le cas du processus scénario, la répartition peut avoir lieu de manière synchrone ou asynchrone, 
avec ou sans partage d'informations entre clients.


\section{Sémantique}\label{sec.semantique}
On décrit ici la sémantique des différentes méthodes de synchronisation vues précédemment.

On fait le choix de décrire ce modèle via la sémantique existante de i-score ; cela peut être ramené à des réseaux de Petri assez facilement.



- implémentation : on rajoute des délais dans les expressions.

Cas synchrone : 
* Sous-expression devient vraie
* Expression fixe une date possible pour le réseau et notifie les autres puis se fixe au bon délai.

Cas asynchrone : 
* Sous-expression devient vraie
* Expression envoie un message aux autres et se lance immédiatement.
Il faut définir sur quelle machine l'expression est validée.

- Cas de la borne max : trigger, bof

Pour l'instant : group leader ? ou bien tous se communiquent l'information et prennent la décision en fonction de cette information ?

- Pour l'édition temps-réel, on doit permettre d'appliquer des "filtres" (par exemple qui vont rajouter une sur-expression, etc)
\subsection{Méchanismes de synchronisation}
- Cas possibles : 

\begin{lstlisting}
C1 ---t--- C3
C2 ---|--- C4
\end{lstlisting}
* Mode free ou synchronisé : pour chaque client ? 

Cas 1 : 
\begin{lstlisting}
C1 ---t--- C3
\end{lstlisting}
C1 libre, C3 libre : dès que t est vrai, un message est envoyé au master qui envoie à C3 le trigger. C1 s'est déjà arrêté.
Voire, ils peuvent se trig eux même en asynchrone.

C1 synchro, C3 synchro : dès que t est vrai, C1 calcule la date minimale à laquelle C3 peut être notifié, envoie le message et fixe ce temps de son côté.

Cas 2 : 
\begin{lstlisting}
C1 ---t--- C3
C2 ---|
\end{lstlisting}

* Synchronisation de la fin

* Synchronisation du début de la suite

En pratique, on n'implémente que la possibilité de synchro / désynchro toute une time node ; dire qu'une granularité plus fine est possible mais que l'intérêt n'apparaît pas en pratique dans les applications (et complexifierait l'UI pour rien).

Question principale : pour un time-node, comment choisit-on sur quelle machine une condition doit être vérifiée ? Possibilité de conditions groupées : faut-il que ce soit interne au formalisme (i.e. une case à cocher) ou externe (on dit explicitement \lstinline|machine1:/truc && machine2:/truc|)

\subsubsection{Cas ou il y a plusieurs machines dans un groupe}
Attributs d'une expression : 
\begin{itemize}
    \item Qui vérifie cette expression : chaque machine individuellement, toutes les machines d'un groupe, n'importe quelle machine d'un groupe 
\end{itemize}

Séparer les expressions et les time-node : notamment pour le cas de la borne max, ou on doit avertir d'autres machines par la suite.

Deux axes : 
\begin{figure}[h]
\begin{tabular}{c|c|c}
\diagbox{Décision}{Sync} & Oui & Non \\
\midrule
Locale & 1 & 2 \\
\midrule
Partagée & 3 & 4\\        
\end{tabular}
\end{figure}
\begin{itemize}
    \item Décision locale : Pour un trigger, soit tout le scénario est local (donc même contrainte avant / après et certains ordis n'exécutent pas du tout ce scénario), soit le premier à avoir un résultat avertit les suivants. -> Le résultat des autres machines n'influence pas l'exécution sur une machine donnée.
    \item Décision partagée : tous les éléments d'un groupe doivent prendre une décision avec une politique donnée : soit tous doivent vérifier la condition, soit un seul.
    \item Décision synchrone : On veut que les éléments s'arrêtent et démarrent le plus proche possible d'une même date en wall clock.
    \item Décision asynchrone : On veut que les éléments s'arrêtent et démarrent le plus vite possible dès qu'une information est disponible.
\end{itemize}

\begin{enumerate}
\item Décision locale synchrone : pas de sens
\item Décision locale asynchrone : pas de sens... 
\item Décision partagée synchrone : toutes les machines attendent qu'un unique choix soit effectué pour le résultat de l'expression
\item Décision partagée asynchrone
\end{enumerate}
Extension via système de composants


On ne synchronise qu'à chaque point d'entrée / de sortie d'un groupe de contraintes / time nodes
\subsection{Extension des possibilités d'écriture}
* Paramètres partagés

* Pattern matching sur addresses 

\section{Évaluation}\label{sec.evaluation}
- Comparaison de l'algorithme "simple" et de l'algorithme avec retard


- Latence : prendre moyenne et écart-type sur les dix dernières valeurs ? Ou juste dernière valeur ?
En LAN gigabit on est en général < à une milliseconde.

\section{Citations}

Toutes les références bibliographiques des citations devront être listées dans la section "\textsc{References}", numérotées et en ordre alphabétique. Toutes les références  listées devront être citées dans le texte. Quand  vous vous référez au document dans le texte, précisez son numéro \cite{Author:00}.

%\begin{thebibliography}{2}

%\end{thebibliography}

\end{document}
